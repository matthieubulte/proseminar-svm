% Created 2017-06-02 Fri 08:59
\documentclass[bigger]{beamer}
\usepackage[utf8]{inputenc}
\usepackage[T1]{fontenc}
\usepackage{fixltx2e}
\usepackage{graphicx}
\usepackage{longtable}
\usepackage{float}
\usepackage{wrapfig}
\usepackage{rotating}
\usepackage[normalem]{ulem}
\usepackage{amsmath}
\usepackage{textcomp}
\usepackage{marvosym}
\usepackage{wasysym}
\usepackage{amssymb}
\usepackage{hyperref}
\tolerance=1000
\usetheme{default}
\author{Matthieu Bulté}
\date{06-06-2017}
\title{Learning with Kernels}
\hypersetup{
  pdfkeywords={},
  pdfsubject={},
  pdfcreator={Emacs 25.2.1 (Org mode 8.2.10)}}
\begin{document}

\maketitle

\section{Learning with Kernels}
\label{sec-1}

\begin{frame}[label=sec-1-1]{Are are we doing?}
\bar{Machine learning} Mathematics is about transforming hard problems
into problems trivial to solve
\end{frame}

\begin{frame}[label=sec-1-2]{What is our problem?}
We have data
We have labels
We want to infer a rule to label new data
\end{frame}


\begin{frame}[label=sec-1-3]{Less abstract please}
Decision function
Decision Boundary
(Plots of many different decision boundaries)
\end{frame}

\begin{frame}[label=sec-1-4]{A trivial problem}
Linearly seperale data set
\end{frame}

\begin{frame}[label=sec-1-5]{Solving the trivial problem}
The Line
\end{frame}

\begin{frame}[label=sec-1-6]{Demo 1 - which line?}
\end{frame}

\begin{frame}[label=sec-1-7]{Margin maximization}
Picking the best trivial solution
(opt problem)
\end{frame}

\begin{frame}[label=sec-1-8]{Demo 1 - cont'd}
\end{frame}

\begin{frame}[label=sec-1-9]{Da dual}
(lagrangian)
\end{frame}

\begin{frame}[label=sec-1-10]{Support Vectors}
H only depends on support vectors!
\end{frame}

\begin{frame}[label=sec-1-11]{A not so trivial problem}
non linearly separable datasets
\end{frame}

\begin{frame}[label=sec-1-12]{Time to be smart}
How do we make this problem easier?
\end{frame}

\begin{frame}[label=sec-1-13]{Transforming the problem}
Space travel

show a pic of a rocket -> 
not this kind of space travel\ldots{}
\end{frame}

\begin{frame}[label=sec-1-14]{Space Travel}
Project the data to another space where the problem is easy
Solve the problem
Bring the easy solution to the hard problem
\end{frame}

\begin{frame}[label=sec-1-15]{Demo 2}
\end{frame}

\begin{frame}[label=sec-1-16]{Not yet there}
Projection is expensive
(polynomial has factorial grows)
\end{frame}

\begin{frame}[label=sec-1-17]{Could we avoid it?}
$\phi$(x) = (x$_{\text{1}}^{\text{2}}$, x$_{\text{2}}^{\text{2}}$, sqrt(2)x$_{\text{1x}}$$_{\text{2}}$)
<$\phi$(x), $\phi$(y)> = \ldots{} = (<x, y>)$^{\text{2}}$ := k(x, y)
\end{frame}

\begin{frame}[label=sec-1-18]{Hello from the other side}
Change our PoV, we have

(phi, space) -> kernel
\end{frame}

\begin{frame}[label=sec-1-19]{Hello from the other side}
Change our PoV, we \bar{have} want  

(phi, space) <- kernel

\alert{mind blown}
\end{frame}

\begin{frame}[label=sec-1-20]{Mercer \& co}
Many different space can be constructed, all based on the same idea.
We chose to present the easy one and left out the useful one. Intuition
is king.
\end{frame}

\begin{frame}[label=sec-1-21]{In other words}
Plug \& Play
\end{frame}

\begin{frame}[label=sec-1-22]{Demo 3 - some kernels}
\url{https://cs.stanford.edu/people/karpathy/svmjs/demo/}
\end{frame}

\begin{frame}[label=sec-1-23]{the good, the bad, \ldots{}}
no uggly :)

bad
\begin{itemize}
\item parameters tuning
\item training time
\item domain knowledge (what do I know when talking about very complex problems?)
\end{itemize}


good
\begin{itemize}
\item E[P(error)] <= \ldots{}
\item performance on small data sets
\item domain knowledge
\end{itemize}
\end{frame}

\begin{frame}[label=sec-1-24]{Thank you}
\end{frame}
% Emacs 25.2.1 (Org mode 8.2.10)
\end{document}