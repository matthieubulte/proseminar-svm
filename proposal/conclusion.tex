\section{Conclusion}

In this paper we have shown following

\begin{itemize}
\item Changing the objective function of the learning problem to maximizing the margin from the hyperplane to the training set not only result in a theoretically better classifier, but also brought us more information about the nature of the classification problem through the support vectors.
\item There exists a sound theory allowing us to change any scalar product based learning algorithm to use a kernelized version, allowing to run the algorithm in a non linear projection of the training set, without having to actually run the projection.
\item Kernels are not only a way to improve performances of the training and classification, but they allow us to incorporate domain knowledge to the classifier in order to improve the performance of the classifier.
\end{itemize}

We have seen the good sides of Support Vector Machines, omitting sometimes to mention some of the downsides of this algorithm. The two main issues one finds with Support Vector Machine is the high cost of the optimization problem being solved. Also, a fair amount of time has to be spent in order to find the proper combination of hyperparameters such as the choice of the kernel but also its parameters, such as the degree of the polynomial kernel.

Thankfully, these issues can be tackled with modern tooling. Modern implementations of Support Vector Machines solve the performance issues for some specific cases, such as the pegasos algorithm \cite{Shalev-Shwartz:2007:PPE:1273496.1273598} for training a linear classifier. Concerning the choice of parameters, one can use methods such as cross validated grid search in order to find combinations of hyper parameters that perform best without overfitting the training set.

Thus, even in the days of deep neural networks, Support Vector Machine stays a relevant learning algorithm. Their comparatively good training performances as well as their low number of hyper parameters tuning and the generalization capacity of the learned classifier makes Support Vector Machines an easy to use algorithm that also provides satisfying results.

%%% Local Variables:
%%% mode: latex
%%% TeX-master: "learning_with_kernels"
%%% End: