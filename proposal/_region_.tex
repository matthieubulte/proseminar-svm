\message{ !name(learning_with_kernels.tex)}%%% Local Variables:
%%% mode: latex
%%% TeX-master: t
%%% End:


%% This LaTeX template is based on the following example file included in the ieeetran
%% package:
%% bare_conf.tex 
%% V1.2
%% 2002/11/18
%% by Michael Shell
%% mshell@ece.gatech.edu
%% (requires IEEEtran.cls version 1.6b or later) with an IEEE conference paper.


% Note that the a4paper option is mainly intended so that authors in
% countries using A4 can easily print to A4 and see how their papers will
% look in print. Authors are encouraged to use U.S. letter paper when 
% submitting to IEEE. Use the testflow package mentioned above to verify
% correct handling of both paper sizes by the author's LaTeX system.
%
% Also note that the "draftcls" or "draftclsnofoot", not "draft", option
% should be used if it is desired that the figures are to be displayed in
% draft mode.
%
% This paper can be formatted using the peerreviewca
% (instead of conference) mode.
\documentclass[conference, a4paper]{IEEEtran-modified}
% If the IEEEtran.cls has not been installed into the LaTeX system files, 
% manually specify the path to it:
% \documentclass[conference]{../sty/IEEEtran} 

\IEEEoverridecommandlockouts

% some very useful LaTeX packages include:

\usepackage{cite}       % Written by Donald Arseneau
                        % V1.6 and later of IEEEtran pre-defines the format
                        % of the cite.sty package \cite{} output to follow
                        % that of IEEE. Loading the cite package will
                        % result in citation numbers being automatically
                        % sorted and properly "ranged". i.e.,
                        % [1], [9], [2], [7], [5], [6]
                        % (without using cite.sty)
                        % will become:
                        % [1], [2], [5]--[7], [9] (using cite.sty)
                        % cite.sty's \cite will automatically add leading
                        % space, if needed. Use cite.sty's noadjust option
                        % (cite.sty V3.8 and later) if you want to turn this
                        % off. cite.sty is already installed on most LaTeX
                        % systems. The latest version can be obtained at:
                        % http://www.ctan.org/tex-archive/macros/latex/contrib/supported/cite/

%\usepackage{graphicx}  % Written by David Carlisle and Sebastian Rahtz
                        % Required if you want graphics, photos, etc.
                        % graphicx.sty is already installed on most LaTeX
                        % systems. The latest version and documentation can
                        % be obtained at:
                        % http://www.ctan.org/tex-archive/macros/latex/required/graphics/
                        % Another good source of documentation is "Using
                        % Imported Graphics in LaTeX2e" by Keith Reckdahl
                        % which can be found as esplatex.ps and epslatex.pdf
                        % at: http://www.ctan.org/tex-archive/info/
% NOTE: for dual use with latex and pdflatex, instead load graphicx like:
\ifx\pdfoutput\undefined
	\usepackage{graphicx}
\else
	\usepackage[pdftex]{graphicx}
\fi

% However, be warned that pdflatex will require graphics to be in PDF
% (not EPS) format and will preclude the use of PostScript based LaTeX
% packages such as psfrag.sty and pstricks.sty. IEEE conferences typically
% allow PDF graphics (and hence pdfLaTeX). However, IEEE journals do not
% (yet) allow image formats other than EPS or TIFF. Therefore, authors of
% journal papers should use traditional LaTeX with EPS graphics.
%
% The path(s) to the graphics files can also be declared: e.g.,
% \graphicspath{{../eps/}{../ps/}}
% if the graphics files are not located in the same directory as the
% .tex file. This can be done in each branch of the conditional above
% (after graphicx is loaded) to handle the EPS and PDF cases separately.
% In this way, full path information will not have to be specified in
% each \includegraphics command.
%
% Note that, when switching from latex to pdflatex and vice-versa, the new
% compiler will have to be run twice to clear some warnings.
\graphicspath{{figures/}}


%\usepackage{psfrag}    % Written by Craig Barratt, Michael C. Grant,
                        % and David Carlisle
                        % This package allows you to substitute LaTeX
                        % commands for text in imported EPS graphic files.
                        % In this way, LaTeX symbols can be placed into
                        % graphics that have been generated by other
                        % applications. You must use latex->dvips->ps2pdf
                        % workflow (not direct pdf output from pdflatex) if
                        % you wish to use this capability because it works
                        % via some PostScript tricks. Alternatively, the
                        % graphics could be processed as separate files via
                        % psfrag and dvips, then converted to PDF for
                        % inclusion in the main file which uses pdflatex.
                        % Docs are in "The PSfrag System" by Michael C. Grant
                        % and David Carlisle. There is also some information 
                        % about using psfrag in "Using Imported Graphics in
                        % LaTeX2e" by Keith Reckdahl which documents the
                        % graphicx package (see above). The psfrag package
                        % and documentation can be obtained at:
                        % http://www.ctan.org/tex-archive/macros/latex/contrib/supported/psfrag/

%\usepackage{subfigure} % Written by Steven Douglas Cochran
                        % This package makes it easy to put subfigures
                        % in your figures. i.e., "figure 1a and 1b"
                        % Docs are in "Using Imported Graphics in LaTeX2e"
                        % by Keith Reckdahl which also documents the graphicx
                        % package (see above). subfigure.sty is already
                        % installed on most LaTeX systems. The latest version
                        % and documentation can be obtained at:
                        % http://www.ctan.org/tex-archive/macros/latex/contrib/supported/subfigure/

%\usepackage{url}       % Written by Donald Arseneau
                        % Provides better support for handling and breaking
                        % URLs. url.sty is already installed on most LaTeX
                        % systems. The latest version can be obtained at:
                        % http://www.ctan.org/tex-archive/macros/latex/contrib/other/misc/
                        % Read the url.sty source comments for usage information.

%\usepackage{stfloats}  % Written by Sigitas Tolusis
                        % Gives LaTeX2e the ability to do double column
                        % floats at the bottom of the page as well as the top.
                        % (e.g., "\begin{figure*}[!b]" is not normally
                        % possible in LaTeX2e). This is an invasive package
                        % which rewrites many portions of the LaTeX2e output
                        % routines. It may not work with other packages that
                        % modify the LaTeX2e output routine and/or with other
                        % versions of LaTeX. The latest version and
                        % documentation can be obtained at:
                        % http://www.ctan.org/tex-archive/macros/latex/contrib/supported/sttools/
                        % Documentation is contained in the stfloats.sty
                        % comments as well as in the presfull.pdf file.
                        % Do not use the stfloats baselinefloat ability as
                        % IEEE does not allow \baselineskip to stretch.
                        % Authors submitting work to the IEEE should note
                        % that IEEE rarely uses double column equations and
                        % that authors should try to avoid such use.
                        % Do not be tempted to use the cuted.sty or
                        % midfloat.sty package (by the same author) as IEEE
                        % does not format its papers in such ways.

\usepackage{amsmath}    % From the American Mathematical Society
                        % A popular package that provides many helpful commands
                        % for dealing with mathematics. Note that the AMSmath
                        % package sets \interdisplaylinepenalty to 10000 thus
                        % preventing page breaks from occurring within multiline
                        % equations. Use:
\interdisplaylinepenalty=2500
                        % after loading amsmath to restore such page breaks
                        % as IEEEtran.cls normally does. amsmath.sty is already
                        % installed on most LaTeX systems. The latest version
                        % and documentation can be obtained at:
                        % http://www.ctan.org/tex-archive/macros/latex/required/amslatex/math/



% Other popular packages for formatting tables and equations include:

%\usepackage{array}
% Frank Mittelbach's and David Carlisle's array.sty which improves the
% LaTeX2e array and tabular environments to provide better appearances and
% additional user controls. array.sty is already installed on most systems.
% The latest version and documentation can be obtained at:
% http://www.ctan.org/tex-archive/macros/latex/required/tools/

% Mark Wooding's extremely powerful MDW tools, especially mdwmath.sty and
% mdwtab.sty which are used to format equations and tables, respectively.
% The MDWtools set is already installed on most LaTeX systems. The lastest
% version and documentation is available at:
% http://www.ctan.org/tex-archive/macros/latex/contrib/supported/mdwtools/


% V1.6 of IEEEtran contains the IEEEeqnarray family of commands that can
% be used to generate multiline equations as well as matrices, tables, etc.


% Also of notable interest:

% Scott Pakin's eqparbox package for creating (automatically sized) equal
% width boxes. Available:
% http://www.ctan.org/tex-archive/macros/latex/contrib/supported/eqparbox/



% Notes on hyperref:
% IEEEtran.cls attempts to be compliant with the hyperref package, written
% by Heiko Oberdiek and Sebastian Rahtz, which provides hyperlinks within
% a document as well as an index for PDF files (produced via pdflatex).
% However, it is a tad difficult to properly interface LaTeX classes and
% packages with this (necessarily) complex and invasive package. It is
% recommended that hyperref not be used for work that is to be submitted
% to the IEEE. Users who wish to use hyperref *must* ensure that their
% hyperref version is 6.72u or later *and* IEEEtran.cls is version 1.6b 
% or later. The latest version of hyperref can be obtained at:
%
% http://www.ctan.org/tex-archive/macros/latex/contrib/supported/hyperref/
%
% Also, be aware that cite.sty (as of version 3.9, 11/2001) and hyperref.sty
% (as of version 6.72t, 2002/07/25) do not work optimally together.
% To mediate the differences between these two packages, IEEEtran.cls, as
% of v1.6b, predefines a command that fools hyperref into thinking that
% the natbib package is being used - causing it not to modify the existing
% citation commands, and allowing cite.sty to operate as normal. However,
% as a result, citation numbers will not be hyperlinked. Another side effect
% of this approach is that the natbib.sty package will not properly load
% under IEEEtran.cls. However, current versions of natbib are not capable
% of compressing and sorting citation numbers in IEEE's style - so this
% should not be an issue. If, for some strange reason, the user wants to
% load natbib.sty under IEEEtran.cls, the following code must be placed
% before natbib.sty can be loaded:
%
% \makeatletter
% \let\NAT@parse\undefined
% \makeatother
%
% Hyperref should be loaded differently depending on whether pdflatex
% or traditional latex is being used:
%
%\ifx\pdfoutput\undefined
%\usepackage[hypertex]{hyperref}
%\else
%\usepackage[pdftex,hypertexnames=false]{hyperref}
%\fi
%
% Pdflatex produces superior hyperref results and is the recommended
% compiler for such use.



% *** Do not adjust lengths that control margins, column widths, etc. ***
% *** Do not use packages that alter fonts (such as pslatex).         ***
% There should be no need to do such things with IEEEtran.cls V1.6 and later.


\usepackage[latin1]{inputenc}
\usepackage{mathtools}
\usepackage{amssymb}
\usepackage{amsmath}
\usepackage{amsbsy}
\usepackage{mathrsfs}
\usepackage{color}
\usepackage{pgfplots}
\usepackage{tikz}
\usetikzlibrary{calc}
\usepackage{relsize}

\pagestyle{empty}
\hyphenation{}

\begin{document}

\message{ !name(kernel_svm.tex) !offset(-25) }
\section{Kernel methods}

The previous section introduced the large margin classifier. The large margin classifier, as we saw, allows us the find the best seperating hyperplane between two linearly seperable classes of a dataset. In this section we introduce the concept of kernels, allowing us to learn non linear decision boudaries, continuing with the Mercer theorem and kernel trick, results from functional analysis allowing us to use the newly introducd ideas of kernels with the previously introduces agorithm, with only light modifications.

\subsection{Non linearly separable data}

Very often, the decision boundary one is trying to learn is in it's nature non-linear. One trick that is often used in order to use linear learning algorithms is to add one or several dimensions to the data points by applying a non linear function of the coordinates of the point, to then learn the separating hyperplane in the projected space. This method is known as features engineering, because we add engineered features to our data points.


This approach would be very simple to implement in our current algorithm. Let $\phi : \mathcal{X} \rightarrow \mathcal{V}$ be a non linear transformation from our original space $\mathcal{X}$ to some higer dimensional Hilbert space $\mathcal{V}$. We can now modify the previously introduced algorithm by replacing simply every scalar product $\left<\cdot , \cdot\right>$ by the scalar product on the projected points in the $\mathcal{V}$ space, namely $\left<\phi(\cdot), \phi(\cdot)\right>$.


In order to better understand kernels and the kernel trick, let's follow the example of polynomial projection of degree $d$, in which every point is projected to a vector containing every monomial of degree $d$. For instance, choosing $d = 2$ together with a $2$ dimensional input space leads to the projection $\phi : \mathbb{R}^2 \rightarrow \mathbb{R}^4$ defined as

\begin{equation*}
    \phi(x_1, x_2) \mapsto (x_1^2, x_2^2, x_1x_2, x_2x_1)
\end{equation*}

This representation, while adding more separation capabilities has the problem that the image feature space grows at an exponential rate together with $d$, making the choice of a larger $d$ prohibitivly expensive in terms of space usage. Though, the margin maximization algorithm only uses the scalar product of the observations in the feature space, and the following equations show that the projection in the high dimensional space is not required to the computation of the scalar product

\begin{equation*}
  \begin{aligned}
    \left<\phi(x), \phi(x')\right>
    &= [x]_1^2[x']_1^2 + [x]_2^2[x']_2^2 + 2[x]_1[x]_2[x']_1[x']_2\\
    &= \left<x, x'\right>^2 \\
    &=: k(x, x')
  \end{aligned}
\end{equation*}

We call $k(\cdot, \cdot) : \mathbb{R}^2 \times \mathbb{R}^2 \rightarrow \mathbb{R}$ the kernel representation of the scalar product in the space $\phi(\mathbb{R}^2)$. One can generalize this idea to any polynomial degree $d$ as shown in \textcolor[rgb]{1,0,0}{ref?}, making the computation of the scalar product in the $d$-th monomial space as trivial as computing the scalar product in the original space.

\subsection{Kernel trick}

We have seen in the previous example, that it is sometimes possible to find a function $k :\mathcal{X} \times \mathcal{X} \rightarrow \mathbb{R}$ with $k(x, x') = \left<\phi(x), \phi(x')\right>$ for some projection $\mathcal{X} \rightarrow \mathcal{V}$. With these special functions, it is possible to then, only lightly modifying our original algorithm, to run our learning algorithm in another vector space without the computational cost of projecting our data in this other space. This is called the kernel trick.

We will now change our point of view, and try to define the properties defining the class of functions being the representation of a scalar product of the projection of its inputs in another vector space. This means, which properties must hold for $k$ in order for a $\phi$ to exist with the correspondance property. \textcolor[rgb]{1,0,0}{better name needed, also a definition would be nice} 

Let $k :\mathcal{X} \times \mathcal{X} \rightarrow \mathbb{R}$ with $k(x, x') = \left<\phi(x), \phi(x')\right>_{\mathcal{V}}$ for some projection $\mathcal{X} \rightarrow \mathcal{V}$, because $\left<\cdot, \cdot\right>_{\mathcal{V}}$ is a scalar product, the following properties must hold

\begin{itemize}
\item \textit{Symmetry}
  \begin{equation*}
    \begin{aligned}
      & \forall y_1, y_2 \in \mathcal{V}.\ 
      \left<y_1, y_2\right>_{\mathcal{V}} = \overline{\left<y_2, y_1\right>_{\mathcal{V}}} & \Rightarrow\\
      & \forall y_1, y_2 \in \phi(\mathcal{X}).\ 
      \left<y_1, y_2\right>_{\mathcal{V}} = \overline{\left<y_2, y_1\right>_{\mathcal{V}}} & \Rightarrow\\
      &\forall x_1, x_2 \in \mathcal{X}.\ 
      \left<\phi(x_1), \phi(x_2)\right>_{\mathcal{V}} = \overline{\left<\phi(x_2), \phi(x_1)\right>_{\mathcal{V}}} &\Rightarrow\\
      &\forall x_1, x_2 \in \mathcal{X}.\ 
         k\left(x_1, x_2\right) = \overline{k\left(x_2, x_1\right)}
    \end{aligned}
  \end{equation*}

\item \textit{Positive definitness} Positive definitness of a kernel is a stronger property than the one required for the positive definiteness in the features space. A kernel is said to be positive definite if for every $\alpha_1, ..., \alpha_n \in \mathbb{R}$ following inequality holds
  \begin{equation*}
    \sum_{i,j=1}^n\alpha_i\alpha_jk\left(x_i, x_j\right) \geq 0
  \end{equation*}
\end{itemize}

We will show by construction, that these properties are sufficient to the proof of the existence of the desired space. The literature contains several examples of vector spaces and projections with the desired property. We will here proove the existence of such a space by construction.

Let $k : \mathcal{X} \times \mathcal{X} \rightarrow \mathbb{R}$ be symmetrical and positive definite, and $\mathbb{R}^{\mathcal{X}}$ be the set of functions from a non empty set $\mathcal{X}$ to $\mathbb{R}$. We define the reproducing kernel map as follow

\begin{equation}
  \begin{aligned}
    \phi : \mathcal{X} \rightarrow \mathcal{H}\\
    x \mapsto k(\cdot, x)
  \end{aligned}
\end{equation}

We now show how the mapped observations of the training set $\left\{k\left(\cdot, x_1\right), ..., k\left(\cdot, x_n\right)\right\}$ spans a Hilbert space in which the reproducing property hold with the scalar defined on the span set $\left<k\left(\cdot, x_i\right), k\left(\cdot, x_j\right)\right> = k\left(x_i, x_j\right)$ and its canonical derivation for the spanned space

\begin{equation*}
  \begin{aligned}
    \left<f, g\right>
    &= \left<\sum_{i=1}^n\alpha_ik(\cdot, x_i), \sum_{j=1}^n\beta_jk(\cdot, x_j)\right>\\
    &= \sum_{i,j=1}^n\alpha_i\beta_j \left<k\left(\cdot, x_i\right), k\left(\cdot, x_j\right)\right> \\
    &= \sum_{i,j=1}^n\alpha_i\beta_j k\left(x_i, x_j\right)
  \end{aligned}
\end{equation*}

Positive definitness, bilinearity and symmetry can all be trivially derived from the definition of this scalar product together with the symmetry and positive definitness of the kernel function. Then attentive reader will notive that the functions $\left\{k\left(\cdot, x_1\right), ..., k\left(\cdot, x_n\right)\right\}$ must nost necessarily be linearly independant, leading to a non unqiue mapping from $f$ to the coefficients $\alpha_1, ..., \alpha_n$. To see that this doesn't imply the ill definitness of the defined scalar product, we note that

\begin{equation*}
  \left<f, g\right> = \sum_{j=1}^n\beta_j f\left(x_j\right)
\end{equation*}

and by symmetry

\begin{equation*}
  \left<f, g\right> = \left<g, f\right> = \sum_{i=1}^n\alpha_i g\left(x_i\right)
\end{equation*}

Which shows that the scalar product does not depend on the choice of coefficients for the functions $f$ and $g$ confirming the scalar product is well defined. Thus, the previously defined scalar product, together with the norm $|f| = \sqrt{\left<f, f\right>}$ define a valid Hilbert space and the correspondance property holds

\begin{equation*}
  \begin{aligned}
    \left<\phi(x_i), \phi(x_j)\right> =\ &\left<k\left(\cdot, x_i\right), k\left(\cdot, x_j\right)\right>\\
    =\ &k(x_i, x_j)
  \end{aligned}
\end{equation*}

Now that we have presented the kernel trick as well as the existence of a corresponding Hilbert space, the next section will focus on finding and creating useful kernels, as well as presenting some classical kernels.

\subsection {Some useful kernels}

We have defined in the last section what kernels are and how we can modify our algorithm to learn non-linear boundaries in our data. What we have omited so far is how is one supposed to choose develop or choose a kernel when facing a practical problem.

In order to answer the previous question, we have to stop looking at kernels as a way to avoid heavy, or sometimes impossible, computations of a scalar product in some feature space. Instead of that, we look at the scalar product as a similarity measurement. Thus, the kernel trick can become a way to find the proper Hilbert space in which one's definition of similarity defines a scalar product, letting us then use the Support Vector Machine machinery as a way to train a classifier based on our definition of similarity.

What do we mean by similarity? Let's first write down a more geometrical  definition of the euclidean scalar product in $\mathbb{R}^n$ 

\begin{equation*}
  \left<x, x'\right> = \|x\|\|x'\|\text{cos}\left(angle\left(x, x'\right)
  \right)
\end{equation*}

This equivalent definition of the euclidean scalar product makes it more visible what the underlying notion of similarity of the euclidean scalar product is measured by the angle between the two vectors, scaled by their length. A similar kernel often used is the cosine similarity kernel, defined as

\begin{equation*}
  k\left(x, x'\right) = \frac{x^Tx'}{\|x\|\|x'\|} = \text{cos}\left(angle\left(x, x'\right)\right)
\end{equation*}

This notion of similarity will thus compare observations by first normalizing them to then compare the directions of the vectors. This similarity measure is often used in the context of text classification, where each entry of the vector corresponds to the number of occurences of a word in a text. The normalization will transform count of occurences in frequencies, thus having a similarity based on the relative importance of each word, independantly of the length of the text.

Cosine similarity is an interesting similarity, but one natural way to reason about similarity of points is their distance from one another. The next kernel we introduce incorporates the notions of distance as a similarity measurement, is called the Radial Basis Function kernel and is defined as followed for some hyperparameter $\sigma \in \mathbb{R}$

\begin{equation*}
  k\left(x, x'\right) = \text{exp}\left(-\frac{\|x - x'\|^2}{\sigma^2}\right)
\end{equation*}

One can 

%%% Local Variables:
%%% mode: latex
%%% TeX-master: "learning_with_kernels"
%%% End:

\message{ !name(learning_with_kernels.tex) !offset(-107) }

\end{document}
%%% Local Variables:
%%% mode: latex
%%% TeX-master: t
%%% End:
