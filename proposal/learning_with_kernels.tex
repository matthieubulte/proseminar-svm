%%% Local Variables:
%%% mode: latex
%%% TeX-master: t
%%% End:


%% This LaTeX template is based on the following example file included in the ieeetran
%% package:
%% bare_conf.tex 
%% V1.2
%% 2002/11/18
%% by Michael Shell
%% mshell@ece.gatech.edu
%% (requires IEEEtran.cls version 1.6b or later) with an IEEE conference paper.


% Note that the a4paper option is mainly intended so that authors in
% countries using A4 can easily print to A4 and see how their papers will
% look in print. Authors are encouraged to use U.S. letter paper when 
% submitting to IEEE. Use the testflow package mentioned above to verify
% correct handling of both paper sizes by the author's LaTeX system.
%
% Also note that the "draftcls" or "draftclsnofoot", not "draft", option
% should be used if it is desired that the figures are to be displayed in
% draft mode.
%
% This paper can be formatted using the peerreviewca
% (instead of conference) mode.
\documentclass[conference, a4paper]{IEEEtran-modified}
% If the IEEEtran.cls has not been installed into the LaTeX system files, 
% manually specify the path to it:
% \documentclass[conference]{../sty/IEEEtran} 

\IEEEoverridecommandlockouts

% some very useful LaTeX packages include:

\usepackage{cite}       % Written by Donald Arseneau
                        % V1.6 and later of IEEEtran pre-defines the format
                        % of the cite.sty package \cite{} output to follow
                        % that of IEEE. Loading the cite package will
                        % result in citation numbers being automatically
                        % sorted and properly "ranged". i.e.,
                        % [1], [9], [2], [7], [5], [6]
                        % (without using cite.sty)
                        % will become:
                        % [1], [2], [5]--[7], [9] (using cite.sty)
                        % cite.sty's \cite will automatically add leading
                        % space, if needed. Use cite.sty's noadjust option
                        % (cite.sty V3.8 and later) if you want to turn this
                        % off. cite.sty is already installed on most LaTeX
                        % systems. The latest version can be obtained at:
                        % http://www.ctan.org/tex-archive/macros/latex/contrib/supported/cite/

%\usepackage{graphicx}  % Written by David Carlisle and Sebastian Rahtz
                        % Required if you want graphics, photos, etc.
                        % graphicx.sty is already installed on most LaTeX
                        % systems. The latest version and documentation can
                        % be obtained at:
                        % http://www.ctan.org/tex-archive/macros/latex/required/graphics/
                        % Another good source of documentation is "Using
                        % Imported Graphics in LaTeX2e" by Keith Reckdahl
                        % which can be found as esplatex.ps and epslatex.pdf
                        % at: http://www.ctan.org/tex-archive/info/
% NOTE: for dual use with latex and pdflatex, instead load graphicx like:
\ifx\pdfoutput\undefined
	\usepackage{graphicx}
\else
	\usepackage[pdftex]{graphicx}
\fi

% However, be warned that pdflatex will require graphics to be in PDF
% (not EPS) format and will preclude the use of PostScript based LaTeX
% packages such as psfrag.sty and pstricks.sty. IEEE conferences typically
% allow PDF graphics (and hence pdfLaTeX). However, IEEE journals do not
% (yet) allow image formats other than EPS or TIFF. Therefore, authors of
% journal papers should use traditional LaTeX with EPS graphics.
%
% The path(s) to the graphics files can also be declared: e.g.,
% \graphicspath{{../eps/}{../ps/}}
% if the graphics files are not located in the same directory as the
% .tex file. This can be done in each branch of the conditional above
% (after graphicx is loaded) to handle the EPS and PDF cases separately.
% In this way, full path information will not have to be specified in
% each \includegraphics command.
%
% Note that, when switching from latex to pdflatex and vice-versa, the new
% compiler will have to be run twice to clear some warnings.
\graphicspath{{figures/}}


%\usepackage{psfrag}    % Written by Craig Barratt, Michael C. Grant,
                        % and David Carlisle
                        % This package allows you to substitute LaTeX
                        % commands for text in imported EPS graphic files.
                        % In this way, LaTeX symbols can be placed into
                        % graphics that have been generated by other
                        % applications. You must use latex->dvips->ps2pdf
                        % workflow (not direct pdf output from pdflatex) if
                        % you wish to use this capability because it works
                        % via some PostScript tricks. Alternatively, the
                        % graphics could be processed as separate files via
                        % psfrag and dvips, then converted to PDF for
                        % inclusion in the main file which uses pdflatex.
                        % Docs are in "The PSfrag System" by Michael C. Grant
                        % and David Carlisle. There is also some information 
                        % about using psfrag in "Using Imported Graphics in
                        % LaTeX2e" by Keith Reckdahl which documents the
                        % graphicx package (see above). The psfrag package
                        % and documentation can be obtained at:
                        % http://www.ctan.org/tex-archive/macros/latex/contrib/supported/psfrag/

%\usepackage{subfigure} % Written by Steven Douglas Cochran
                        % This package makes it easy to put subfigures
                        % in your figures. i.e., "figure 1a and 1b"
                        % Docs are in "Using Imported Graphics in LaTeX2e"
                        % by Keith Reckdahl which also documents the graphicx
                        % package (see above). subfigure.sty is already
                        % installed on most LaTeX systems. The latest version
                        % and documentation can be obtained at:
                        % http://www.ctan.org/tex-archive/macros/latex/contrib/supported/subfigure/

%\usepackage{url}       % Written by Donald Arseneau
                        % Provides better support for handling and breaking
                        % URLs. url.sty is already installed on most LaTeX
                        % systems. The latest version can be obtained at:
                        % http://www.ctan.org/tex-archive/macros/latex/contrib/other/misc/
                        % Read the url.sty source comments for usage information.

%\usepackage{stfloats}  % Written by Sigitas Tolusis
                        % Gives LaTeX2e the ability to do double column
                        % floats at the bottom of the page as well as the top.
                        % (e.g., "\begin{figure*}[!b]" is not normally
                        % possible in LaTeX2e). This is an invasive package
                        % which rewrites many portions of the LaTeX2e output
                        % routines. It may not work with other packages that
                        % modify the LaTeX2e output routine and/or with other
                        % versions of LaTeX. The latest version and
                        % documentation can be obtained at:
                        % http://www.ctan.org/tex-archive/macros/latex/contrib/supported/sttools/
                        % Documentation is contained in the stfloats.sty
                        % comments as well as in the presfull.pdf file.
                        % Do not use the stfloats baselinefloat ability as
                        % IEEE does not allow \baselineskip to stretch.
                        % Authors submitting work to the IEEE should note
                        % that IEEE rarely uses double column equations and
                        % that authors should try to avoid such use.
                        % Do not be tempted to use the cuted.sty or
                        % midfloat.sty package (by the same author) as IEEE
                        % does not format its papers in such ways.

\usepackage{amsmath}    % From the American Mathematical Society
                        % A popular package that provides many helpful commands
                        % for dealing with mathematics. Note that the AMSmath
                        % package sets \interdisplaylinepenalty to 10000 thus
                        % preventing page breaks from occurring within multiline
                        % equations. Use:
\interdisplaylinepenalty=2500
                        % after loading amsmath to restore such page breaks
                        % as IEEEtran.cls normally does. amsmath.sty is already
                        % installed on most LaTeX systems. The latest version
                        % and documentation can be obtained at:
                        % http://www.ctan.org/tex-archive/macros/latex/required/amslatex/math/



% Other popular packages for formatting tables and equations include:

%\usepackage{array}
% Frank Mittelbach's and David Carlisle's array.sty which improves the
% LaTeX2e array and tabular environments to provide better appearances and
% additional user controls. array.sty is already installed on most systems.
% The latest version and documentation can be obtained at:
% http://www.ctan.org/tex-archive/macros/latex/required/tools/

% Mark Wooding's extremely powerful MDW tools, especially mdwmath.sty and
% mdwtab.sty which are used to format equations and tables, respectively.
% The MDWtools set is already installed on most LaTeX systems. The lastest
% version and documentation is available at:
% http://www.ctan.org/tex-archive/macros/latex/contrib/supported/mdwtools/


% V1.6 of IEEEtran contains the IEEEeqnarray family of commands that can
% be used to generate multiline equations as well as matrices, tables, etc.


% Also of notable interest:

% Scott Pakin's eqparbox package for creating (automatically sized) equal
% width boxes. Available:
% http://www.ctan.org/tex-archive/macros/latex/contrib/supported/eqparbox/



% Notes on hyperref:
% IEEEtran.cls attempts to be compliant with the hyperref package, written
% by Heiko Oberdiek and Sebastian Rahtz, which provides hyperlinks within
% a document as well as an index for PDF files (produced via pdflatex).
% However, it is a tad difficult to properly interface LaTeX classes and
% packages with this (necessarily) complex and invasive package. It is
% recommended that hyperref not be used for work that is to be submitted
% to the IEEE. Users who wish to use hyperref *must* ensure that their
% hyperref version is 6.72u or later *and* IEEEtran.cls is version 1.6b 
% or later. The latest version of hyperref can be obtained at:
%
% http://www.ctan.org/tex-archive/macros/latex/contrib/supported/hyperref/
%
% Also, be aware that cite.sty (as of version 3.9, 11/2001) and hyperref.sty
% (as of version 6.72t, 2002/07/25) do not work optimally together.
% To mediate the differences between these two packages, IEEEtran.cls, as
% of v1.6b, predefines a command that fools hyperref into thinking that
% the natbib package is being used - causing it not to modify the existing
% citation commands, and allowing cite.sty to operate as normal. However,
% as a result, citation numbers will not be hyperlinked. Another side effect
% of this approach is that the natbib.sty package will not properly load
% under IEEEtran.cls. However, current versions of natbib are not capable
% of compressing and sorting citation numbers in IEEE's style - so this
% should not be an issue. If, for some strange reason, the user wants to
% load natbib.sty under IEEEtran.cls, the following code must be placed
% before natbib.sty can be loaded:
%
% \makeatletter
% \let\NAT@parse\undefined
% \makeatother
%
% Hyperref should be loaded differently depending on whether pdflatex
% or traditional latex is being used:
%
%\ifx\pdfoutput\undefined
%\usepackage[hypertex]{hyperref}
%\else
%\usepackage[pdftex,hypertexnames=false]{hyperref}
%\fi
%
% Pdflatex produces superior hyperref results and is the recommended
% compiler for such use.



% *** Do not adjust lengths that control margins, column widths, etc. ***
% *** Do not use packages that alter fonts (such as pslatex).         ***
% There should be no need to do such things with IEEEtran.cls V1.6 and later.


\usepackage[latin1]{inputenc}
\usepackage{mathtools}
\usepackage{amssymb}
\usepackage{amsmath}
\usepackage{amsbsy}
\usepackage{mathrsfs}
\usepackage{color}
\usepackage{pgfplots}

\pagestyle{empty}
\hyphenation{}

\begin{document}

\title{Learning with Kernels}

\author{
\authorblockN{Matthieu Bult�}
\authorblockA{Fakult�t f�r Mathematik\\Technische Universit�t M�nchen\\
Email: matthieu.bulte@tum.de} 
}

\specialpapernotice{Proseminar Data Mining}

\maketitle


\textcolor[rgb]{1,0,0}{add to large margin intuition section that larger margin also reduces the space of possible hyperplanes}

\textcolor[rgb]{1,0,0}{use jabref for refs}

\textcolor[rgb]{1,0,0}{figure out where we need a vector space and where we need hilbert space}



\begin{abstract}
  We present an algorithm for finding the observations relevant to classification of observations in two or more classes: the Support Vector Machone. The support observations are then used in the decision function as a linear combination, and their number allows us to derive an approximation of the classifier's performance. The kernel trick is then presented, a generic method appliable to scalar product-based algorithm is shown to enable to let one incorporate domain knowledge about the classification problem in order to improve classification performance.
\end{abstract}

%%% Local Variables:
%%% mode: latex
%%% TeX-master: "learning_with_kernels"
%%% End:


\begin{keywords}
  support vector machines, machine learning, statistics
\end{keywords}

\section{Introduction}

Machine learning can be seen as the science of automatically finding regularities or patterns in data. 

%%% Local Variables:
%%% mode: latex
%%% TeX-master: "learning_with_kernels"
%%% End:

%%% Local Variables:
%%% mode: latex
%%% TeX-master: "learning_with_kernels"
%%% End:

\section {First steps: hard margin classifier}

The hard margin classifier learns the parameters $\mathbf{w} \in \mathbb{R}^n$ and $b \in \mathbb{R}$ of a hyperplane $\mathscr{H} \subset \mathbb{R}^n$, separating two classes of observations $\mathbf{x}$ of dimension $n$. To train it, the algorithm takes a set of $m$ observations $\mathbf{x_i}$ together with a vector of labels $y_i = \pm 1$ for each of the observations:

$$
(\mathbf{x_1}, y_1), (\mathbf{x_2}, y_2), \dotsc, (\mathbf{x_m}, y_m)
$$

The learned parameters are then used to define a decision function $f$, assigning to points of $\mathbb{R}^n$ a label $\pm 1$ corresponding to the side of the hyperplane on which the points stands:

\begin{equation}
  f(\mathbf{x}) = sgn(\langle\mathbf{w}, \mathbf{x}\rangle + b)
\end{equation}

\subsection {Margin maximization}

One particularity about the learning algorithm used for Support Vector Machines is the loss function the algorithm tries to optimize. Other learning algorithms typically chose a loss function based on the empirical risk, defined as following for a function $f$ and a training set $(\mathbf{x}, \mathbf{y})$ of size $m$:

\begin{equation}
  R_{emp}[f] = \frac{1}{m}\sum^m_{i=0}\frac{1}{2}|f(\mathbf{x_i}) - y_i|
\end{equation}

The margin maximization algorithm instead, searches within the space of hyperplanes properly classifying the training set, for the hyperplane with the largest distance from the training points to the hyperplane. This can be translated in the following constrained optimization problem:

\begin{equation}
  \begin{aligned}
    &\underset{\mathbf{w} \in \mathbb{R}^n, b \in \mathbb{R}} {\text{maximize}}
    & & M := \frac{1}{\|\mathbf{w}\|} \underset{i} {min}\ 
    y_i(\langle\mathbf{w},\mathbf{x_i}\rangle + b)\\
    &\text{subject to}
    & &y_i(\langle\mathbf{w},\mathbf{x_i}\rangle + b) \ge M
  \end{aligned}
\end{equation}

Assuming that a solution to this optimization problem exisits, it still has the issue that although the solution hyperplane is unique, there is an inifity of parameter vectors $\|\mathbf{w}\|$ resulting in the solution hyperplane, only differing in their length. Uniqueness of $\mathbf{w}$ can be ensured by adding to its length the follwoing contraint:

$$M\|\mathbf{w}\| = 1$$

We can thus reformulate the optimization problem into the following quadratic optimization problem which is known to have a numerical solution:

\begin{equation}
  \begin{aligned}
    &\underset{\mathbf{w} \in \mathbb{R}^n, b \in \mathbb{R}} {\text{minimize}}
    & &\|\mathbf{w}\|\\
    &\text{subject to}
    & &y_i(\langle\mathbf{w},\mathbf{x_i}\rangle + b) \ge 1
  \end{aligned}
\end{equation}

\subsection {Support vectors}

Although the previous formulation of the problem helped us to better understand the ideas of support vector machines, the resulting quadratic problem is in practice too hard to solve for a very large $m$. We will, in this section, introduce another formulation of the optimization problem that will not only make it computationally more practical, but will also give us more insights about the resulting hyperplane.



\section {Kernel methods}

\subsection {The kernel trick}
\subsection {Some useful kernels}
\section{Conclusion}

In this paper we have shown following

\begin{itemize}
\item Changing the objetive function of the learning problem to maximizing the margin from the hyperplane to the training set not only result in a theoretically better classifier, but also brought us more information about the nature of the classification problem through the support vectors.
\item There exists a sound theory allowing us to change any scalar product based learning algorithm to use a kernelized version, allowing to run the algorithm in a non linear projection of the training set, without having to actually run the projection.
\item  Kernels are not only a way to improve performances of the training and classification, but they allow us to incorporate domain knowledge to the classifier in order to improve the performance of the classifier.
\end{itemize}

We have seen the good sides of Support Vector Machines, omiting sometimes to mention some of the downsides of this algorithm. The two main issues one finds with Support Vector Machine is the high cost of the optimization problem being solved. Also, a fair amount of time has to be spent in order to find the proper combination of hyperparameters such as the choice of the kernel but also its parameters, such as the degree of the polynomial kernel.

Thankfully, these issues can be tackled with modern tooling. Modern implementations of Support Vector Machines solve the performance issues for some specific cases, such as the pegasos algorithm \textcolor[rgb]{1,0,0}{ref?} for training a linear classifier. Concerning the choice of parameters, one can use methods such as cross validated grid search in order to find combinations of hyper parameters that perform best without overfitting the training set.

Thus, even in the days of deep neural networks, Support Vector Machine stays a relevant learning algorithm. Their comparatively good training performances as well as their low number of hyper parameters tuning and the generalization capacity of the learned classifier makes Support Vector Machines an easy to use algorithm that also provides satisfying resulsts.

%%% Local Variables:
%%% mode: latex
%%% TeX-master: "learning_with_kernels"
%%% End:

\bibliographystyle{IEEEtran}

\bibliography{IEEEabrv,references}

\end{document}